\documentclass[12pt]{article}
\usepackage[utf8]{inputenc}
\usepackage[T1]{fontenc}
\usepackage[english]{babel}
\usepackage{amsmath, amssymb}
\usepackage{graphicx}
\usepackage{booktabs}
\usepackage{geometry}
\usepackage{tabularx}
\usepackage{caption}
\usepackage[colorlinks=true, linkcolor=blue, citecolor=blue, urlcolor=blue]{hyperref}
\usepackage{float}
\usepackage{enumitem}
\usepackage{setspace}
\usepackage[absolute,overlay]{textpos} % Pour positionner librement
\setlength{\TPHorizModule}{1mm}
\setlength{\TPVertModule}{1mm}
\geometry{margin=1in}
\setstretch{1.2}

\title{Wasserstein distance-based clustering \\ \large Machine Learning - Optimal Transport}
\author{Enzo MORAN}
\date{}

\begin{document}

% Positionnement du logo en haut à droite (ajuste les coordonnées si besoin)
\begin{textblock*}{30mm}(170mm,10mm) % largeur bloc 30mm, position (x=170mm,y=10mm)
    \includegraphics[width=25mm]{ensae.png}
\end{textblock*}

\maketitle

\tableofcontents

\newpage

\section*{Introduction}
\addcontentsline{toc}{section}{Introduction}

The objective of this project is to model a macroeconomic variable as a time series and to forecast its short-term evolution using the software \texttt{R} (the full code is included in the file \texttt{gsimonnet\_emoran.zip}).

\section{K-means}

\subsection{Yubo Zhuang et al.(2022) definitions of K-means}




\section{title}

\subsection{title}

\subsubsection{title}




\appendix

\section{Appendix A: ...}
\label{appendix:label}


\section{Appendix B: ...}
\label{appendix:label}


\end{document}


--------------------------------------------------------------------------------
FIGURES

------------------------------
introduire la figure graph.png :
------------------------------
\begin{figure}[H]
\centering
\includegraphics[width=0.8\textwidth]{graph.png}
\caption{title}
\label{fig:graph}
\end{figure}

------------------------------
mettre 2 images cote à cote :
------------------------------
\begin{figure}[H]
\centering
\begin{minipage}{0.48\textwidth}
  \centering
  \includegraphics[width=\linewidth]{residuals_acf.png}
  \caption{ACF of the residuals}
  \label{fig:acf_residuals}
\end{minipage}
\hfill
\begin{minipage}{0.48\textwidth}
  \centering
  \includegraphics[width=\linewidth]{residuals_distribution.png}
  \caption{Distribution of the residuals compared to a Gaussian density}
  \label{fig:density_residuals}
\end{minipage}
\end{figure}

------------------------------
mettre une table :
------------------------------
\begin{table}[H]
\centering
\caption{title}
\label{tab:label}
\small
\makebox[\textwidth][c]{%
\begin{tabular}{lll}
\toprule
Coefficient & Estimate & Std. Error \\
\midrule
AR(1) & -0.3197 & 0.0483 \\
\bottomrule
\end{tabular}
}
\normalsize
\end{table}

------------------------------
mettre une équation :
------------------------------
\[
\mathbb{E}[X_{T+1} \mid \mathcal{F}_T, Y_{T+1}] = a_{11} X_T + a_{12} Y_T + \rho \frac{\sigma_X}{\sigma_Y} \cdot \varepsilon_{Y,T+1},
\]


--------------------------------------------------------------------------------
LateX : TEXTE

------------------------------
faire référence à une figure : Figure~\ref{fig:graph}
------------------------------
mettre en gras : \textbf{txt}
------------------------------
citer un fichier : \texttt{txt}
------------------------------
mettre des maths au milieu du texte : \( Y_{T+1} \)
------------------------------
sauter une grosse ligne : \vspace{0.5\baselineskip}
------------------------------
mettre un hyperlien : \href{https://lien}{texte}
------------------------------

--------------------------------------------------------------------------------
Equations 

------------------------------
mettre un gros espace entre 2 equations sur la même ligne : \quad
------------------------------
chapeau pour des grandes lettres : \widehat{}
------------------------------
équivalence : \iff
------------------------------
barre verticale : \mid
------------------------------
faire des matrices ou des vecteurs : 
\begin{pmatrix}
a_{11} & a_{12} \\
a_{21} & a_{22}
\end{pmatrix}
------------------------------
aligner ses équations : 
\begin{align*}
a &= b \\
a &= a
\end{align*}
